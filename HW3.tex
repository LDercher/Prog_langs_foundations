\documentclass[10pt]{article}


  \usepackage{ dsfont }
  \usepackage{amsmath,amssymb,amsthm}
  
  
  
  \pdfpagewidth   8.5in
  
  \pdfpageheight    11in
  
  \setlength\paperheight {11in}
  
  \setlength\paperwidth  {8.5in}
  
  
  
  %Adjust the margins
  
  \topmargin        -0.5in
  
  \oddsidemargin      0in
  
  \evensidemargin     0in
  
  \textheight       9.0in
  
  \textwidth        6.5in
  
  
  
  \setlength{\parindent}{0pt}
  
  \setlength{\parskip}{5pt}
  
  
  
  \newcommand{\squash}{\itemsep=0pt\parskip=0pt}
  
  \newcommand{\squish}{\topsep=0pt\partopsep=0pt}
  
  \newcommand{\sample}[1]{\begin{center}\framebox{\parbox{0.8\textwidth}{\texttt{#1}}}\end{center}}
  
  \newcommand{\tr}{\ensuremath{\rightsquigarrow}}
  
  
  
  
  
  \theoremstyle{definition}
  
  \newtheorem{defn}{Definition}
  
  \newtheorem{prob}{Problem}
  
  \newtheorem{bp}{Textbook Problem}
  \newtheorem{ts}{Textbook Solution}
  
  
  
  \begin{document}
  
  
  
  \noindent
  
  \textbf{Homework 3: Domains and Denotations} \hfill \emph{Due: September 24, 2018, 9:00 AM}
  
  \hrule
  
  
  
  \vspace{.3in}
  
  
  
  Section 2.10.3 exercises 1--2.
  
  
  
  \begin{bp}
  Determine if the following equations are monotonic and continuous
  \begin{equation}
  f(n) = \{ 0 \textrm{ if } n \in \mathds{N} \textrm{ is even, } 1 \textrm{ if } n \in \mathds{N} \textrm{ is odd, } \omega \textrm{ if } n = \omega
  \end{equation}
  
  \begin{equation}
  f(n) = \{ 2*n \textrm{ if } n \in \mathds{N} \textrm{ , } \omega \textrm{ if } n = \omega 
  \end{equation}
  
  \end{bp}
  
  \begin{ts}
  
  1. This function is not monotonic nor is it continuous. Why? ex. $f(4) = 0$, $f(3) = 1$. In this case we have $x < y$ but not $f(x) < f(y)$. The function is not continuous since it can only output 0, 1, or $\omega$.
  
  2.This function is monotonic. If we have an $x < y$, $2*x < 2*y$, therefore $f(x) < f(y)$. This function is also continuous. This is because for all n, f(n) has a value, and no skips or jumps take place in the function at any time.  
  \end{ts}
  
  
  
  Section 2.10.4 exercise 1.
  
  \begin{bp}
  
  consider while command while $ x \neq y $ do $x = x + 1 $. Write out first three approximations.
  \end{bp}
  
  \begin{ts}
  the first approximation is the following
  
  $\bot_f (\phi) = \bot$
  
  second:
  
  $F(\bot_f (\phi)) = \{ \phi$ if $\phi(x) = \phi(y)$, $\bot_f(\phi[x := x + 1]) = \bot$ otherwise
  
  third:
  
  $F(F(\bot_f (\phi))) = \{ \phi$ if $\phi(x) = \phi(y)$, $\bot_f(\phi[x := x + 1]) = \bot$ if x = y - 1, $\bot_f(\phi[x := [x + 1] + 1]) = \bot$ otherwise
  
  
  
  \end{ts}
  
  
  
  Section 2.11.3 exercises 1 and 2 ($\star$)
  
  \begin{ts}
  
  (a) elements of $ B \mapsto A$ are as follows
  
  (0,1),(1,0),(0,2),(2,0)(1,2),(2,1),(0,0),(1,1),(2,2)
  
  (b) The elements of $ B \mapsto A$ that are related by pointwise ordering are the following
  
  A(0,1) $\mapsto$ B(0,1), A(0,1) $\mapsto$ B(0,2), A(1,0) $\mapsto$ B(1,0), A(1,0) $\mapsto$ B(2,0)
  
  (c) the elements that are monotonic functions from $(B,\subseteq_B)$ to $(A,\subseteq_A)$ are the following
  
  $\{0 \mapsto 0, 1 \mapsto 1\}$, $\{0 \mapsto 0, 1 \mapsto 2\}$, $\{0 \mapsto 1, 1 \mapsto 1\}$, $\{0 \mapsto 1, 1 \mapsto 2\}$, $\{0 \mapsto 2, 1 \mapsto 1\}$, $\{0 \mapsto 2, 1 \mapsto 2, \}$
  
  (d) A monotonic function from $B \mapsto A$ to $B \mapsto A$ is the following
  
  
  
  
  \end{ts}
  
  
  \end{document}
  
  