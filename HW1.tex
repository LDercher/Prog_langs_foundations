\documentclass[10pt]{article}

  \usepackage{amsmath,amssymb,amsthm}
  
  \pdfpagewidth   8.5in
  \pdfpageheight    11in
  \setlength\paperheight {11in}
  \setlength\paperwidth  {8.5in}
  
  %Adjust the margins
  \topmargin        -0.5in
  \oddsidemargin      0in
  \evensidemargin     0in
  \textheight       9.0in
  \textwidth        6.5in
  
  \newcommand{\squash}{\itemsep=0pt\parskip=0pt}
  \newcommand{\squish}{\topsep=0pt\partopsep=0pt}
  \newcommand{\sample}[1]{\begin{center}\framebox{\parbox{0.8\textwidth}{\texttt{#1}}}\end{center}}
  \newcommand{\tr}{\ensuremath{\rightsquigarrow}}
  
  
  \theoremstyle{definition}
  \newtheorem{defn}{Definition}
  \newtheorem{prob}{Problem}
  \newtheorem{sol}{Solution}
  
  \begin{document}
  
  \noindent
  \textbf{Homework 1: Functions, Relations, and $FO(\mathbb Z)$} \hfill \emph{Due: September 4, 2018, 9:00 AM}
  \hrule
  
  \vspace{.3in}
  
  \section*{Functions}
  
  \begin{defn}
    A function $f : A \to B$ is:
    \begin{enumerate}
    \item \emph{Injective} if whenever $f(x) = f(y)$, then $x = y$;
    \item \emph{Surjective} if for every $y$ there is some $x$ such that $f(x) = y$;
    \item \emph{Bijective} if it is both injective and surjective; and,
    \item \emph{Invertable} if there is some function $g : B \to A$ such that $g \circ f = I_A$ and
      $f \circ g = I_B$ (where $I_X$ is the identity function from $X$ to $X$).
    \end{enumerate}
  \end{defn}
  
  \begin{prob}
    Restate Definitions 1.1, 1.2, and 1.4 using the syntax of $FO(\mathbb Z)$.
  \end{prob}
  
  \begin{sol}
    \begin{equation}
        \forall xy.f(x) = f(y) \Rightarrow x = y
    \end{equation}
      \begin{equation}   
      \forall y. \exists x. f(x) = y
      \end{equation}
      \begin{equation}
        \forall xy. f(x) = y \Leftrightarrow g(y) = x
      \end{equation}
  \end{sol}
  
  \begin{prob}
    Suppose that $f$ has two inverses $g$ and $h$.  (That is, each of $g$ and $h$ meet the conditions
    in Definition 1.4.)  Show that $g = h$.
  \end{prob}
  
  \begin{sol}
      \begin{align}
      h \circ f \circ g &= h \circ f \circ g \\
    I_B \circ g &= h \circ I_B \\
      g &= h 
      \end{align}
  \end{sol}
  
  \begin{prob}[$\star$]
    Show that if $f$ has an inverse, then $f$ is bijective.
  \end{prob}
  
  \begin{sol}
   proof: if f has an inverse, then it is surjective
   \begin{align}
   Let:  f: A \rightarrow B, f^{-1}: B \rightarrow A, a \in A, b \in B \\
   f^{-1}(b) &= a \\
   f(a) &= f(f^{-1}(b)) \\
       &= f \circ f^{-1} (b) \\
        &= I_B \circ b \\
        &= b   
   \end{align}
   proof: if f has an inverse, then it is injective
   \begin{align}
    Let: a_1, a_2 \in A, f(a_1) = f(a_2), x = f^{-1}(y), y = f(x_1) \\
    a_2 &= I_A(a_2) \\
        &= f^{-1} \circ f (a_2) \\
        &= f^{-1}(b) \\
        &= a
   \end{align}
   
   \begin{align}
   a_1 &= I_A(a_1) \\
        &= f^{-1} \circ f (a_1) \\
        &= f^{-1}(b) \\
        &= a
   \end{align}
   
   Therefore $a_1$ = $a_2$
  \end{sol}
  
  \section*{Relations}
  \newcommand\R{\mathrel{R}}
  
  
  \begin{defn}
    A relation $R$ is:
    \begin{enumerate}
    \item \emph{Reflexive} if for every $x$, $x \R x$;
    \item \emph{Symmetric} if whenever $x \R y$, then $y \R x$;
    \item \emph{Transitive} if whenever $x \R y$ and $y \R z$, then $x \R z$.
    \end{enumerate}
    A relation that is reflexive, symmetric, and transitive is called an \emph{equivalence} relation.
  \end{defn}
  
  \begin{prob}
    Show (or give a counterexample) that if $R$ and $S$ are equivalence relations, then so is $R \cap
    S$.
  \end{prob}
  
  \begin{sol}
   \begin{align}
   Assume: {(x,y),(y,z)} \in R \cap S \\
     \Rightarrow (x,y) \cap (y,z) \in R,S \\
     \Rightarrow (x,z) \in R,S \\
     \Rightarrow (x,z) \in R \cap S
   \end{align}
  \end{sol}
  
  \begin{prob}
    Show (or give a counterexample) that if $R$ and $S$ are equivalence relations, then so is
    $R \cup S$.
  \end{prob}
  
  \begin{sol}
  \begin{align}
    Fix: R &= \big\{(x,x),(x,y),(y,y)\big\} \\
    S &= \big\{(x,x),(y,x),(y,y)\big\} \\
      R \cup S  &= \big\{(x,x),(y,y)\big\}
  \end{align}
  This case is not transitive, therefore $R \cup $S is not transitive
  \end{sol}
  
  
  \section*{$FO(\mathbb Z)$}
  
  Section 1.13.1 exercises 1--3.
  
  
  
  
  \end{document}
  