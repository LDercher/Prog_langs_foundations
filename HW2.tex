\documentclass[10pt]{article}

\usepackage{amsmath,amssymb,amsthm,enumitem}

\pdfpagewidth   8.5in
\pdfpageheight    11in
\setlength\paperheight {11in}
\setlength\paperwidth  {8.5in}

%Adjust the margins
\topmargin        -0.5in
\oddsidemargin      0in
\evensidemargin     0in
\textheight       9.0in
\textwidth        6.5in

\setlength{\parindent}{0pt}
\setlength{\parskip}{5pt}

\newcommand{\squash}{\itemsep=0pt\parskip=0pt}
\newcommand{\squish}{\topsep=0pt\partopsep=0pt}
\newcommand{\sample}[1]{\begin{center}\framebox{\parbox{0.8\textwidth}{\texttt{#1}}}\end{center}}
\newcommand{\tr}{\ensuremath{\rightsquigarrow}}


\theoremstyle{definition}
\newtheorem{defn}{Definition}
\newtheorem{prob}{Problem}
\newtheorem{ts}{Textbook Solution}


\begin{document}

\noindent
\textbf{Homework 2: $FO(Z)$ and \texttt{while}} \hfill \emph{Due: September 7, 2018, 9:00 AM}
\hrule

\vspace{.3in}

\section*{$FO(\mathbb Z)$}

Section 1.14.1 exercise 1-2.

\begin{ts}
 Every two numbers has a common divisor in FO(Z)
 \begin{equation}
 	\forall x. \forall y. \exists w. ( x \textpipe w) \wedge (y \textpipe w)
 \end{equation}
\end{ts}

Section 1.14.2 exercise 1.

Section 1.14.3 exercise 2.

Section 1.14.4 exercise 2 ($\star$)

\begin{prob}
  Using the following definitions of addition and multiplication:
  \begin{align*}
    0 + y &= y & 0 \times y &= 0 \\
    Sx + y &= S(x + y) & Sx \times y &= y + x \times y
  \end{align*}
  Show the following:
  \begin{enumerate}[label=(\alph*)]
  \item $\forall x y z. x \times z + y \times z = (x + y) \times z$
  \item $\forall x y z. x \times (y \times z) = (x \times y) \times z$
  \end{enumerate}
\end{prob}

\section*{The \texttt{while} language}

Section 2.10.1 exercises 2-5.

\end{document}
