\documentclass[10pt]{article}





  \usepackage{ dsfont }

  \usepackage{amsmath,amssymb,amsthm}

  

  

  

  \pdfpagewidth   8.5in

  

  \pdfpageheight    11in

  

  \setlength\paperheight {11in}

  

  \setlength\paperwidth  {8.5in}

  

  

  

  %Adjust the margins

  

  \topmargin        -0.5in

  

  \oddsidemargin      0in

  

  \evensidemargin     0in

  

  \textheight       9.0in

  

  \textwidth        6.5in

  

  

  

  \setlength{\parindent}{0pt}

  

  \setlength{\parskip}{5pt}

  

  

  

  \newcommand{\squash}{\itemsep=0pt\parskip=0pt}

  

  \newcommand{\squish}{\topsep=0pt\partopsep=0pt}

  

  \newcommand{\sample}[1]{\begin{center}\framebox{\parbox{0.8\textwidth}{\texttt{#1}}}\end{center}}

  

  \newcommand{\tr}{\ensuremath{\rightsquigarrow}}

  

  

  

  

  

  \theoremstyle{definition}

  

  \newtheorem{defn}{Definition}

  

  \newtheorem{prob}{Problem}

  

  \newtheorem{bp}{Textbook Problem}

  \newtheorem{ts}{Textbook Solution}

  

  

  

  \begin{document}

  

  

  

  \noindent

  

  \textbf{Homework 3: Domains and Denotations} \hfill \emph{author: Luke Dercher}

  

  \hrule

  

  

  

  \vspace{.3in}

  

  

  

  Section 2.10.3 exercises 1--2.

  

  

  

  \begin{bp}

  Determine if the following equations are monotonic and continuous in the following domain $(\mathds{N} \cup {\omega}, \leq_\omega)$

  \begin{equation}
  f(n) = \{ 0 \textrm{ if } n \in \mathds{N} \textrm{ is even, } 1 \textrm{ if } n \in \mathds{N} \textrm{ is odd, } \omega \textrm{ if } n = \omega
  \end{equation}

  

  \begin{equation}
  f(n) = \{ 2*n \textrm{ if } n \in \mathds{N} \textrm{ , } \omega \textrm{ if } n = \omega 
  \end{equation}

\begin{equation}
    f(n) = \omega
\end{equation}
  

  \end{bp}

  

  \begin{ts}

  

  1. This function is not monotonic nor is it continuous. Why? ex. $f(4) = 0$, $f(3) = 1$. In this case we have $x \leq y$ but not $f(x) \leq f(y)$. The function is not continuous since it can only output 0, 1, or $\omega$.

  

  2.This function is monotonic. If we have an $x \leq y$, $2*x \leq 2*y$, therefore $f(x) \leq f(y)$. This function is also continuous. This is because for all n, f(n) has a value, and no skips or jumps take place in the function at any time.  

  3. This function is monotonic since the input is always mapped to the bottom element. We can observe for any or $x \leq y$, $f(x) \leq f(y)$ since $\omega \leq \omega$ is true. This function is also continuous since it is just a straight line. 
  \end{ts}

  
\begin{bp}
    Determine if the following equations are monotonic and continuous in the following domain $(\mathds{N},|, 1)$
    
    \begin{equation}
        f(n) = n + 1
    \end{equation}
    
    \begin{equation}
    f(n) = \{ n/2 \textrm{ if n is even, } n*2 \textrm{ if n is odd}
    \end{equation}
    
    \begin{equation}
        f(n) = \{ 0 \textrm{ if n is even, } 1 \textrm{ if n is odd}
    \end{equation}
  
\end{bp}
  
\begin{ts}
1. This function is monotonic since we can take any $x \leq y$ and $x + 1 \leq y + 1$ still holds true. This function is not continuous across it's domain of integers related by the divides relation. Take example n = 1, f(n = 1) = 2. We cannot find a value fro n such that f(n) = 1, therefore this function is not continuous over the divides relation of integers domain. 

2. This function is non-monotonic over the domain. Consider counterexample to proof that it is monotonic x = 3 and y = 6. f(x) = 6 and f(y) = 3. Therefore $f(x) \not \leq f(y)$ and hence this function is non-monotonic over the domain. This function is also continuous over the domain. Again by example take n = n/2 if even and n = n+1 / 2 if odd. f(n = even) = n / 4 and f(n = odd) = n + 1. Since both of these functions are continuous we can say for every n there is a corresponding f(n) that outputs that n. 

3. This function is non-monotonic. Take example y = 5 if even and x = 4 if odd. f(y = 5) = 0 and f(x = 4) = 1. Here we have a case of $x \leq y$ but not $f(x) \leq f(y)$. This function is also non-continuous. Observe that it can only map to 0 and 1 which doesn't span across our domain of integeres related by divides. 
\end{ts}

  

  Section 2.10.4 exercise 1.

  

  \begin{bp}

  

  consider while command while $ x \neq y $ do $x = x + 1 $. Write out first three approximations.

  \end{bp}

  

  \begin{ts}

  the first approximation is the following

  

  $\bot_f (\phi) = \bot$

  

  second:

  

  $F(\bot_f (\phi)) = \{ \phi$ if $\phi(x) = \phi(y)$, $\bot_f(\phi[x := x + 1]) = \bot$ otherwise

  

  third:

  

  $F(F(\bot_f (\phi))) = \{ \phi$ if $\phi(x) = \phi(y)$, $\bot_f(\phi[x := x + 1]) = \bot$ if x = y - 1, $\bot_f(\phi[x := [x + 1] + 1]) = \bot$ otherwise

  

  

  

  \end{ts}

  

  

  

  Section 2.11.3 exercises 1 and 2 ($\star$)

  \begin{bp}

	let $(A, \sqsubseteq_A,\bot_A)$ be s.t. \\
	\indent $A = {0,1,2}$ \\ 
	$\forall a \in A. 0 \sqsubseteq_A a$ \\
	$1 \sqsubseteq_A 2$ \\
	$\forall a \in A. a \sqsubseteq_A a$ \\
	$\bot_A = 0$ \\ \\
	
	let $(B, \sqsubseteq_B,\bot_B)$ be s.t. \\
	\indent $B = {0,1}$ \\ 
	$\forall b \in B. 0 \sqsubseteq_B b$ \\
	$\forall b \in B. b \sqsubseteq_B b$ \\
	$\bot_B = 0$
  
  \end{bp}
  

  \begin{ts}

  

  (a) elements of $ B \mapsto A$ are as follows

  

  (0,1),(1,0),(0,2),(2,0)(1,2),(2,1),(0,0),(1,1),(2,2)

  

  (b) The elements of $ B \mapsto A$ that are related by pointwise ordering are the following

  

  A(0,1) $\mapsto$ B(0,1), A(0,1) $\mapsto$ B(0,2), A(0,0) $\mapsto$ B(0,0), A(1,1) $\mapsto$ B(1,1),A(1,1) $\mapsto$ B(2,1),A(1,1) $\mapsto$ B(2,2). 
  $ \forall a_1.a_2. (a_1 = 0) \wedge (a_2 \leq 1) $
  

  (c) the elements that are monotonic functions from $(B,\subseteq_B)$ to $(A,\subseteq_A)$ are the following
    
    $ \forall a_1.a_2. (a_1 = 0) \wedge (a_2 \leq 1) $
    
    the elements that are not are the ones that satisfy the following condition.
    
    $ \forall a_1.a_2. (a_1 > 0) \vee (a_2 \not \leq 1) $

  

  (d) A monotonic function from $B \mapsto A$ to $B \mapsto A$ is the following
    function $B \mapsto A = (m,n) \mapsto (n,m)$. Function $B \mapsto A, B \mapsto A= \{(m,n) \mapsto \{(m,n) \mapsto (n,m)\} \mapsto (n,m)\}$
  
  (e) The least fixed point of the previously described function is (0,0) since 0 is the bottom element in both sets A and B

  \end{ts}
  
  \begin{bp}
      prove by induction on n that while 0 = 0 do x:= x + 1 never terminates
  \end{bp}
  
  \begin{ts}
  base case: n = 1, since 0 = 0 we cannot terminate so $F(1)(\phi) = \bot$ \\
  Assume for integers up to n $F(n)(\phi) = \bot$ (strong induction) \\
  for n + 1 we still have 0 = 0 and hence $F(n+1)(\phi) = \bot$
  \end{ts}

  

  

  \end{document}

  

  