\documentclass[10pt]{article}

\usepackage{amsmath,amssymb,amsthm,mathwidth, textcomp}

\pdfpagewidth   8.5in
\pdfpageheight    11in
\setlength\paperheight {11in}
\setlength\paperwidth  {8.5in}

%Adjust the margins
\topmargin        -0.5in
\oddsidemargin      0in
\evensidemargin     0in
\textheight       9.0in
\textwidth        6.5in

\setlength{\parindent}{0pt}
\setlength{\parskip}{5pt}

\newcommand{\squash}{\itemsep=0pt\parskip=0pt}
\newcommand{\squish}{\topsep=0pt\partopsep=0pt}
\newcommand{\sample}[1]{\begin{center}\framebox{\parbox{0.8\textwidth}{\texttt{#1}}}\end{center}}
\newcommand{\tr}{\ensuremath{\rightsquigarrow}}


\theoremstyle{definition}
\newtheorem{defn}{Definition}
\newtheorem{prob}{Problem}
\newtheorem{bp}{book problem}
\newtheorem{sol}{solution}

\begin{document}

\noindent
\textbf{Homework 6: $\lambda$-calculus} \hfill \emph{Due: December 5, 2018, 9:00 AM}
\hrule

\vspace{.3in}

5.8.5, exercise 2.

\begin{bp}
	Consider the following term

	($\lambda$x.$\lambda$y.x) ($\lambda$z.z) (($\lambda$x.x)($\lambda$y.y))

	(a) For this term, show both the left-to-right and right-to-left call-by-value reduction sequences which end in values, using underlining notation.

	(b) Write down the contexts used for each step in those reduction sequences, and confirm that all contexts are accepted by the appropriate grammer in the chapter.

\end{bp}

\begin{sol}
	left-to-right \newline
	\indent \indent \underline{($\lambda$x.$\lambda$y.x) ($\lambda$z.z)} (($\lambda$x.x)($\lambda$y.y)) \newline
	$\leadsto$   ($\lambda$y.$\lambda$z.z) (\underline{($\lambda$x.x)($\lambda$y.y)}) \newline
	$\leadsto$  \underline{($\lambda$y.$\lambda$z.z)($\lambda$y.y)} \newline
	$\leadsto$ $\lambda$z.z \newline

	right-to-left \newline
	\indent \indent ($\lambda$x.$\lambda$y.x) ($\lambda$z.z) \underline{(($\lambda$x.x)($\lambda$y.y))} \newline
	$\leadsto$   \underline{($\lambda$x.$\lambda$y.x) ($\lambda$z.z)}($\lambda$y.y) \newline
	$\leadsto$ $ [\lambda y.y/ \lambda z.z] \underline{(\lambda y. \lambda z.z)} ( \lambda z.z) $ \newline
	$\leadsto$ $\lambda$z.z \newline




\end{sol}

5.9, exercise 2.

\begin{bp}
	for the purposes of this problem define  t\textdownarrow t' to means $\exists t''.t \leadsto* t'' \wedge t' \leadsto * t'' $ where $\leadsto$ is normal-order reduction (where the outersmost $\beta$-redex is reduced, and reduction proceeds under $\lambda$-binders). As usual, variables can be safely renamed, so that we consider $\lambda x.x$ equivalent to $\lambda y.y$, for example.

	For which of the following terms do we have t\textdownarrow$\lambda x.\lambda y.y$? Please indicated all the terms which satisfy this property.

	(a) $\lambda x. \lambda y.y$ \newline
	(b) $(\lambda x.x) \lambda y.y$ \newline
	(c) $( \lambda x.x x)( \lambda y.y) \lambda x. \lambda y.y$ \newline
	(d) $ \lambda x.( \lambda x. \lambda y.y y)( \lambda x.x)$ \newline
	(e) $ \lambda x.\lambda x.(\lambda y.y)x$
\end{bp}

\begin{sol}.\newline
	(a) $\lambda x. \lambda y.y$ \newline
	sol. If we safely rename x to z we have $\lambda z. \lambda y.y$ and therefore we have t\textdownarrow t' \newline
	(b) $(\lambda x.x) \lambda y.y$ \newline
	sol. This does not reduce to the form $\lambda x.\lambda y.y$ and therefore does not have the property t\textdownarrow t' \newline
	(c) $( \lambda x.x x)( \lambda y.y) \lambda x. \lambda y.y$ \newline
	sol. reducing to normal form we have. \newline
	\underline{$( \lambda x.x x)( \lambda y.y) \lambda x. \lambda y.y$} \newline
	$\leadsto$ $\underline{(( \lambda y.y) \lambda x. \lambda y.y)}(( \lambda y.y) \lambda x. \lambda y.y)$ \newline
	$\leadsto$ $\underline{(\lambda x. \lambda y.y)(( \lambda y.y) \lambda x. \lambda y.y)}$ \newline
	$\leadsto$ $\lambda x. \lambda y.y$ \newline
	again if we safely rename x to z we have $\lambda z. \lambda y.y$ and therefore we have t\textdownarrow t'
	(d) $ \lambda x.( \lambda x. \lambda y.y y)( \lambda x.x)$ \newline
	sol. reducing to normal form we have. \newline
	$ \lambda x.\underline{( \lambda x. \lambda y.y y)( \lambda x.x)}$ \newline
	$\leadsto \lambda x.(\lambda y.y y)$ \newline
	We cannot reduce this any further and hence this does not reduce to the form $\lambda x.\lambda y.y$ and therefore does not have the property t\textdownarrow t' \newline
	(e) $ \lambda x.\lambda x.(\lambda y.y)x$ \newline
	sol. This is already in normal for and hence  This does not reduce to the form $\lambda x.\lambda y.y$ and therefore does not have the property t\textdownarrow t' \newline


\end{sol}

6.8.1, exercises 2--3.

\begin{bp}

	(2) Write a function $add-components$ that takes in a pair $(x,y)$ and returns the pair $x + y$

	(3) Write a function $swap-pair$ that takes in a pair $(x,y)$ and returns the pair $(y,x)$

\end{bp}

\begin{sol}. \newline

	(2) $\lambda p.p( \lambda xy.x)(S(p \lambda x y.y))$

	(3) $\lambda p. \lambda f. f( p \lambda xy.y)(p \lambda xy.x)$

\end{sol}

6.8.2, exercises 2--3.

\begin{bp}. \newline
	(2)Suppose we wish to encode a data type consisting of basic colors $red$ and $blue$, with possible repeated modifier $light$. So example data elements are: $red$, $blue$, $light blue$, $light(light red)$, etc. \newline
	give definitionsfor the three constructors, $red$, $blue$, and $light$, using the Scott encoding.

	(3) Give definitions using the scott encoding for constructors $node$ and $leaf$ for a datatype of binary trees, with data stored at the nodes but not the leaves. So a tree like this, (insert picture of three with values only in internal nodes) $(node 1 leaf(node 2 leaf leaf))$
\end{bp}

\begin{sol}. \newline
	(2) \newline
	red = $\lambda r. \lambda b. \lambda l. r$ \newline
	blue = $\lambda r. \lambda b. \lambda l. b$ \newline
	light = $\lambda c. \lambda r. \lambda b. \lambda l. lc$ \newline
	(3) \newline
	node = $\lambda f. \lambda v. \lambda l. f v l l$ \newline
	leaf = $ \lambda l.l$



\end{sol}

6.9.2, exercise 1.

\begin{bp}
	One way to write a function $eqnaut$ to test whether two Peano numbers are equal is to remove a successor from each of them, until either both numbers are zero, in which case we return $true$; or else one is zero, and the other is not, in which we return $false$.
	(a) Based on this idea, define $eqnuat$ using recursive equations.
	(b) Translate your encoding into a lamda term using the Scott encoding.
\end{bp}

\begin{sol}
  (a) eqnaut Z Z = true \newline
      eqnaut Sm Z = false \newline
      eqnuat z Sn = false \newline
      eqnuat Sm Sn = eqnaut m n \newline
  (b) eqnaut = $ fix ( \lambda eqnaut'. \lambda m n. m ( \lambda m'.n( \lambda n'. eqnaut' m' n')(false)))(n( \lambda n'. false) (true)) )$
\end{sol}

\begin{prob}
	Given the $\lambda$-encoding of lists by:
	\[
		nil = \lambda f. \lambda z. z \qquad cons = \lambda x. \lambda xs. \lambda f. \lambda z. f\,x\,(xs\,f\,z)
	\]
	Show encodings of the $head$ and $tail$ functions satisfying the following equations
	\[
		head\,(cons\,x\,xs) = x \qquad tail\,(cons\,x\,xs) = xs
	\]
	It may help you to recall the approach used to encode the predecessor function.
\end{prob}

\begin{sol}. \newline
	head = $\lambda ys.ys ( \lambda x y.x)(False)$ \newline

	tail = 
\end{sol}

\begin{prob}
	Given the $\lambda$-encoding of the natural numbers by:
	\[
		zero = \lambda f. \lambda x. x \qquad succ = \lambda n. \lambda f. \lambda x. f \, (n\,f\,x)
	\]
	We have the following two definitions of addition:
	\[
		add = \lambda m.\lambda n.\lambda f.\lambda x.m\,succ\,n\,f\,x \qquad add' = \lambda m.\lambda n.\lambda f.\lambda x. m\,f\,(n\,f\,x)
	\]
	Show (by natural number induction) that, for all encodings of natural numbers $m$ and $n$, we have $add\,m\,n = add'\,m\,n$.
\end{prob}

\begin{sol}. \newline
	I will begin by replacing succ function with it's definition and reducing to normal form. \newline
	\[
		add = \lambda m.\lambda n.\lambda f.\lambda x.m \underline{(\lambda n. \lambda f. \lambda x. f \, (n\,f\,x))n\,f\,x}
		\leadsto \lambda m.\lambda n.\lambda f.\lambda x. m\,f\,(n\,f\,x)
	\]
	now I consider the base case for induction.
	add 0 n and add' 0 n
	we have

	add 0 n = \newline
	$ \underline{ \lambda m.\lambda n.\lambda f.\lambda x. m\,f\,(n\,f\,x) (\lambda f. \lambda x. x) n } $ \newline
	$\leadsto \underline{ \lambda n. \lambda f. \lambda x. n f (nfx)(\lambda f. \lambda x.x)}$ \newline
	$ \leadsto \underline{\lambda f. \lambda x. ( \lambda f . \lambda x.x) f n f x}$ \newline
	$ \leadsto \underline{\lambda x. ( \lambda f. \lambda x.x) x n x} $  \newline
	$ \leadsto \underline{(\lambda f. \lambda x. x) x n }$ \newline
	$ \leadsto n$ \newline

	add' 0 n = \newline
	$ \underline{ \lambda m.\lambda n.\lambda f.\lambda x. m\,f\,(n\,f\,x) (\lambda f. \lambda x. x) n } $ \newline
	$\leadsto \underline{ \lambda n. \lambda f. \lambda x. n f (nfx)(\lambda f. \lambda x.x)}$ \newline
	$ \leadsto \underline{\lambda f. \lambda x. ( \lambda f . \lambda x.x) f n f x}$ \newline
	$ \leadsto \underline{\lambda x. ( \lambda f. \lambda x.x) x n x} $  \newline
	$ \leadsto \underline{(\lambda f. \lambda x. x) x n }$ \newline
	$ \leadsto n$ \newline
	Now I will assume add m n = add' m n for m \newline
	And now I will show add Sm n = add' Sm n \newline
	add Sm n = \newline
	$ \underline{\lambda m.\lambda n.\lambda f.\lambda x. m\,f\,(n\,f\,x)( \lambda s. \lambda z s (m s z)) n }$ \newline
	$ \leadsto \underline{\lambda n. \lambda f. \lambda x. n f(n f x)( \lambda s. \lambda z. s (m s z))}$ \newline
	$ \leadsto \underline{ \lambda f. \lambda x. ( \lambda s. \lambda z. s (m s z))f n f x} $ \newline
	$ \leadsto \underline{ \lambda x. ( \lambda s. \lambda z. s (m s z))x n x} $ \newline
	$ \leadsto \underline{ ( \lambda s. \lambda z. s (m s z)x n)} $ \newline
	$ \leadsto x( m x n)$ \newline

	add' Sm n = \newline
	$ \underline{\lambda m.\lambda n.\lambda f.\lambda x. m\,f\,(n\,f\,x)( \lambda s. \lambda z s (m s z)) n }$ \newline
	$ \leadsto \underline{\lambda n. \lambda f. \lambda x. n f(n f x)( \lambda s. \lambda z. s (m s z))}$ \newline
	$ \leadsto \underline{ \lambda f. \lambda x. ( \lambda s. \lambda z. s (m s z))f n f x} $ \newline
	$ \leadsto \underline{ \lambda x. ( \lambda s. \lambda z. s (m s z))x n x} $ \newline
	$ \leadsto \underline{ ( \lambda s. \lambda z. s (m s z)x n)} $ \newline
	$ \leadsto x( m x n)$ \newline

	and hence we can say add m n = add' m n




\end{sol}

\begin{prob}
	$(\star)$ A \emph{snoc-list} is a cons-list in reverse: the sequence $1, 2, 3$ is represented
	\[
		cons\,1\,(cons\,2\,(cons\,3\,nil))
	\]
	as a cons-list, and
	\[
		snoc\,(snoc\,(snoc\,lin\,1)\,2)\,3
	\]
	as a snoc-list.  Give a $\lambda$-encoding of snoc-lists; that is, give $\lambda$ definitions for $lin$, $snoc$, and $foldl$ such that
	\[
		foldl\,f\,z\,(snoc\,(\dots(snoc\,lin\,x_1)\dots)\,x_n) = f\,(\dots(f\,z\,x_1)\dots)\,x_n
	\]
\end{prob}


\end{document}