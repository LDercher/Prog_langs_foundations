\documentclass[10pt]{article}

\usepackage{amsmath,amssymb,amsthm,enumitem}
\usepackage{ stmaryrd,tipa }


\pdfpagewidth   8.5in
\pdfpageheight    11in
\setlength\paperheight {11in}
\setlength\paperwidth  {8.5in}

%Adjust the margins
\topmargin        -0.5in
\oddsidemargin      0in
\evensidemargin     0in
\textheight       9.0in
\textwidth        6.5in

\setlength{\parindent}{0pt}
\setlength{\parskip}{5pt}

\newcommand{\squash}{\itemsep=0pt\parskip=0pt}
\newcommand{\squish}{\topsep=0pt\partopsep=0pt}
\newcommand{\sample}[1]{\begin{center}\framebox{\parbox{0.8\textwidth}{\texttt{#1}}}\end{center}}
\newcommand{\tr}{\ensuremath{\rightsquigarrow}}


\theoremstyle{definition}
\newtheorem{defn}{Definition}
\newtheorem{prob}{Problem}
\newtheorem{ts}{Textbook Solution}


\begin{document}

\noindent
\textbf{Homework 2: $FO(Z)$ and \texttt{while}} \hfill \emph{Due: September 7, 2018, 9:00 AM}
\hrule

\vspace{.3in}

\section*{$FO(\mathbb Z)$}

Section 1.14.1 exercise 1-2.

\begin{ts}
 Every two numbers has a common divisor in FO(Z)
 \begin{equation}
 	\forall x. \forall y. \exists w. ( x \mid w) \wedge (y \mid w)
 \end{equation}
 
 \begin{equation}
 	\forall x. \forall y. \exists z. x \mid z \wedge y \mid z \wedge \forall w. x \mid w \wedge y \mid w \Rightarrow z \leq w
 \end{equation}
\end{ts}

Section 1.14.2 exercise 1.

\begin{ts}

\begin{align*}
	\llbracket \phi \wedge \phi \rrbracket \sigma &=  \llbracket \phi \rrbracket \sigma \\ \llbracket \phi \rrbracket \sigma \wedge \llbracket \phi \rrbracket \sigma &=  \llbracket \phi \rrbracket \sigma \\
\end{align*}
 Take \textphi to equal True or False. True \textturnv  True = True. False \textturnv  False = False
 
 \begin{align*}
	\llbracket \phi + 0 \rrbracket \sigma &=  \llbracket \phi \rrbracket \sigma \\ \llbracket \phi \rrbracket \sigma + 0 &=  \llbracket \phi \rrbracket \sigma \\
\end{align*}
Take \textphi to equal any integer x, add x + 0 = 0 by the meaning of adding nothing to something. In this case x. 

\end{ts}

Section 1.14.3 exercise 2.

\begin{ts}

\begin{align*}
\forall w \forall x. \forall y \forall z. w > x \wedge y > z &\Rightarrow w + y > x + z \\ \forall w \forall x. \forall y \forall z. w > x \wedge y > z &\Rightarrow w + y > x + z
\end{align*}

We can say x + z \textless x + y by left and right monotonicity of addition
We can say x + z \textless x + y \textless w + y by transitive property of \textless
\end{ts}

Section 1.14.4 exercise 2 ($\star$)
\begin{ts}
Use the following properties of addition to prove the following theoroms.
	\begin{enumerate}
	\item 0 + y = y  
	\item S(x)  = S(x + y) 
	\end{enumerate}
	
	\begin{equation}
	\forall x. x + 0 = x
	\end{equation}
	By property 1 of addition, any number y + 0 = y
	
	\begin{equation}
	\forall x. \forall y. x + S(y) = S(x+y)
	\end{equation}
	Take x to be any integer and take S(y) to be any function on y in FO(z), By proprty 2 of addition we can see that any function plus an integer is the function plus the integer
	
	\begin{equation}
	\forall x. \forall y. x + y = y + x
	\end{equation}
\end{ts}



\begin{prob}
  Using the following definitions of addition and multiplication:
  \begin{align*}
    0 + y &= y & 0 \times y &= 0 \\
    Sx + y &= S(x + y) & Sx \times y &= y + x \times y
  \end{align*}
  Show the following:
  \begin{enumerate}[label=(\alph*)]
  \item $\forall x y z. x \times z + y \times z = (x + y) \times z$
  \item $\forall x y z. x \times (y \times z) = (x \times y) \times z$
  \end{enumerate}

\end{prob}

\section*{The \texttt{while} language}

Section 2.10.1 exercises 2-5.

\end{document}
