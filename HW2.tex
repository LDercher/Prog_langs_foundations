\documentclass[10pt]{article}
\usepackage{chngcntr}
\usepackage{amsmath,amssymb,amsthm,enumitem}
\usepackage{ stmaryrd,tipa }

\theoremstyle{definition}
\newtheorem{defn}{Definition}
\newtheorem{prob}{Problem}
\newtheorem{sol}{Solution}
\newtheorem{bp}{Textbook Problem}
\newtheorem{ts}{Textbook Solution}

\counterwithin*{equation}{sol}
\counterwithin*{equation}{ts}
\counterwithin*{equation}{bp}
\pdfpagewidth   8.5in
\pdfpageheight    11in
\setlength\paperheight {11in}
\setlength\paperwidth  {8.5in}

%Adjust the margins
\topmargin        -0.5in
\oddsidemargin      0in
\evensidemargin     0in
\textheight       9.0in
\textwidth        6.5in

\setlength{\parindent}{0pt}
\setlength{\parskip}{5pt}

\newcommand{\squash}{\itemsep=0pt\parskip=0pt}
\newcommand{\squish}{\topsep=0pt\partopsep=0pt}
\newcommand{\sample}[1]{\begin{center}\framebox{\parbox{0.8\textwidth}{\texttt{#1}}}\end{center}}
\newcommand{\tr}{\ensuremath{\rightsquigarrow}}





\begin{document}

\noindent
\textbf{Homework 2: $FO(Z)$ and \texttt{while}} \hfill \emph{Author: Luke Dercher}
\hrule

\vspace{.3in}

\section*{$FO(\mathbb Z)$}

Section 1.14.1 exercise 1-2.

\begin{ts}
 Every two numbers has a common divisor in FO(Z)
 \begin{equation}
 	\forall x. \forall y. \exists w. ( x \mid w) \wedge (y \mid w)
 \end{equation}
 
 \begin{equation}
 	\forall x. \forall y. \exists z. x \mid z \wedge y \mid z \wedge \forall w. x \mid w \wedge y \mid w \Rightarrow z \leq w
 \end{equation}
\end{ts}

Section 1.14.2 exercise 1.

\begin{sol}

\begin{align*}
	\llbracket \phi \wedge \phi \rrbracket \sigma &=  \llbracket \phi \rrbracket \sigma \\ \llbracket \phi \rrbracket \sigma \wedge \llbracket \phi \rrbracket \sigma &=  \llbracket \phi \rrbracket \sigma \\
\end{align*}
 Take \textphi to equal True or False. True \textturnv  True = True. False \textturnv  False = False
 
 \begin{align*}
	\llbracket \phi + 0 \rrbracket \sigma &=  \llbracket \phi \rrbracket \sigma \\ \llbracket \phi \rrbracket \sigma + 0 &=  \llbracket \phi \rrbracket \sigma \\
\end{align*}
Take \textphi to equal any integer x, add x + 0 = 0 by the meaning of adding nothing to something. In this case x. 

\end{sol}

Section 1.14.3 exercise 2.

\begin{sol}

\begin{align*}
\forall w \forall x. \forall y \forall z. w > x \wedge y > z &\Rightarrow w + y > x + z \\ \forall w \forall x. \forall y \forall z. w > x \wedge y > z &\Rightarrow w + y > x + z
\end{align*}

We can say x + z \textless x + y by left and right monotonicity of addition
We can say x + z \textless x + y \textless w + y by transitive property of \textless
\end{sol}

Section 1.14.4 exercise 2 ($\star$)
\begin{sol}
Use the following properties of addition to prove the following theoroms.
	\begin{enumerate}
	\item 0 + y = y  
	\item S(x)  = S(x + y) 
	\end{enumerate}
	
	\begin{equation}
	\forall x. x + 0 = x
	\end{equation}
	By property 1 of addition, any number y + 0 = y
	
	\begin{equation}
	\forall x. \forall y. x + S(y) = S(x+y)
	\end{equation}
	Take x to be any integer and take S(y) to be any function on y in FO(z), By proprty 2 of addition we can see that any function plus an integer is the function plus the integer
	
	\begin{equation}
	\forall x. \forall y. x + y = y + x
	\end{equation}
\end{sol}



\begin{prob}
  Using the following definitions of addition and multiplication:
  \begin{align*}
    0 + y &= y & 0 \times y &= 0 \\
    Sx + y &= S(x + y) & Sx \times y &= y + x \times y
  \end{align*}
  Show the following:
  \begin{enumerate}[label=(\alph*)]
  \item $\forall x y z. x \times z + y \times z = (x + y) \times z$
  \item $\forall x y z. x \times (y \times z) = (x \times y) \times z$
  \end{enumerate}

\end{prob}

\begin{sol}

case x = S(a) 
 
\begin{align*}
(IH) a*z + y * z &= (a + y) * z \\
(def *) (z + a*z) + y* z &= (Sa + y) * z \\
(assoc) z + (a*z + y*z) &= (Sa + y) * z \\
(IH) z + ((a+y)*z) &= (Sa = y) * z \\
(def *) S(a+y) * z &= (Sa + y) * z \\
(def +) (Sa+y) * z &= (Sa + y) * z \\
(Sa + y) * z &= (Sa + y) * z
\end{align*}

case y = S(a) 
 
\begin{align*}
(IH) x*z + a * z &= (x + a) * z \\
(def *) (z + a*z) + x* z &= (Sa + x) * z \\
(assoc) z + (a*z + x*z) &= (Sa + x) * z \\
(IH) z + ((a+x)*z) &= (Sa + x) * z \\
(def *) S(a+x) * z &= (Sa + x) * z \\
(def +) (Sa+x) * z &= (Sa + x) * z \\
(Sa + x) * z &= (Sa + x) * z
\end{align*}
	
\end{sol}


\begin{sol}

case x = S(a) 
 
\begin{align*}
(IH) a*z + y * z &= (a + y) * z \\
(def *) (z + a*z) + y* z &= (Sa + y) * z \\
(assoc) z + (a*z + y*z) &= (Sa + y) * z \\
(IH) z + ((a+y)*z) &= (Sa = y) * z \\
(def *) S(a+y) * z &= (Sa + y) * z \\
(def +) (Sa+y) * z &= (Sa + y) * z \\
(Sa + y) * z &= (Sa + y) * z
\end{align*}

case y = S(a) 
 
\begin{align*}
(IH) x*z + a * z &= (x + a) * z \\
(def *) (z + a*z) + x* z &= (Sa + x) * z \\
(assoc) z + (a*z + x*z) &= (Sa + x) * z \\
(IH) z + ((a+x)*z) &= (Sa + x) * z \\
(def *) S(a+x) * z &= (Sa + x) * z \\
(def +) (Sa+x) * z &= (Sa + x) * z \\
(Sa + x) * z &= (Sa + x) * z
\end{align*}
	
\end{sol}

\begin{sol}

case x = S(a)

\begin{align*}
(IH) a * (y * z) &= ((a * y) * z \\
prove  Sa * (y * z) &= (Sa * y) * z \\
(def *) (y*z) + a (y*z) &= (Sa * y) * z \\
(assoc) z(y + ay) &= (Sa * y) * z\\
(def *)  z * (Sa * y) &= (Sa * y) * z \\
(Sa * y) * z &= (Sa * y) * z
\end{align*}

\end{sol}

\section*{The \texttt{while} language}

Section 2.10.1 exercises 2-5.


\begin{bp}
2.10.1.1: Write a while command that sets $z$ to max of $x$ and $y$
\end{bp}


\begin{ts}
if x - y $>$ 0 then x else if x = y then -1 else y
\end{ts}


\begin{bp}
2.10.1.2: Write a while command that sets $z$ to $x^y$ (assume $y$ is non-neg)
\end{bp}


\begin{ts}
while y > 0;
	z := x * z;
	y := y - 1;
\end{ts}


\begin{bp}
2.10.1.3: What is the meaning of $x$ := $y$; ; $y$:= $z$; $z$:= $x$ in in state { x maps to 0, y maps to 2, z maps to 1}
\end{bp}


\begin{ts}
This means
x := 2;
y := 1;
z := 0
\end{ts}


\begin{bp}
2.10.1.4: What is the meaning of if x $>$ 0 then z := y - x else z := y + x in state {x maps to 3, y maps to 2, z maps to 1}?
\end{bp}


\begin{ts}
since x maps to 3, it is greater than 0. Therefor z is set to the value of y - x (2- (-3)) = -5
\end{ts}


\begin{bp}
Write down the meaning of x := y; y := x in some arbitrary state $ \sigma $
\end{bp}


\begin{ts}
\begin{equation}
\sigma \mapsto \sigma [ x \mapsto y]
\end{equation}
\end{ts}


\end{document}
